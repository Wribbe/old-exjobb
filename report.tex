\documentclass[a4paper,11pt]{article}

\usepackage[utf8]{inputenc}
\usepackage[T1]{fontenc}
\usepackage[english]{babel}
\usepackage{graphicx}
\usepackage{lipsum}

\usepackage{xcolor}
\usepackage{tikz}
\usetikzlibrary{calc}

\usepackage{mathptmx}
\usepackage{helvet}
\usepackage[hidelinks=true]{hyperref}

\usepackage{calc}

\usepackage[a4paper,top=4.9cm, bottom=4.9cm, left=4.05cm,%
  right=4.05cm]{geometry}

% Change table of contents to match.
\usepackage{titletoc}
\usepackage[titles]{tocloft}
\setlength{\cftsubsecindent}{0.3cm}
\setlength{\cftsubsecindent}{0.3cm}
\setlength{\cftsubsubsecindent}{0.6cm}
\addto\captionsenglish{\renewcommand{\contentsname}{Table of contents}}
\renewcommand{\cftdot}{}
\renewcommand{\cftsecfont}{\normalfont\large}
\renewcommand{\cftsecpagefont}{\normalfont\large}
\renewcommand{\cftsubsecpagefont}{\normalfont\large}
\renewcommand{\cftsubsecfont}{\normalfont\large}
\renewcommand{\cftsubsubsecpagefont}{\normalfont\large}
\renewcommand{\cftsubsubsecfont}{\normalfont\large}

\setlength{\parindent}{0cm}
\newcommand\YUGE{\fontsize{24}{12}\selectfont}

%Set title formats.
\usepackage{titlesec}
\newcommand\HOne{\fontsize{20}{12}\selectfont}
\newcommand\HTwo{\fontsize{14}{8}\selectfont}
\titleformat{\section}
  {\normalfont\HOne}
  {\thesection:}
  {0.3cm}
  {}
  {}%
\titleformat{name=\section,numberless}
  {\normalfont\HOne}
  {}
  {0cm}
  {}
  {}%
\titleformat{\subsection}{\normalfont\HTwo}{\thesubsection:}{0.3cm}{}
\def\sectionSpacing{1.5cm}
\titlespacing{\section}{0cm}{10pt}{\sectionSpacing}
\titlespacing{name=\section, numberless}{0cm}{0cm}{\sectionSpacing}
\titlespacing{\subsection}{0cm}{10pt}{\sectionSpacing-0.7cm}
\titlespacing{\subsubsection}{0pt}{10pt}{12pt}

% Reference settings.
\usepackage[autostyle]{csquotes}
\usepackage[
  backend=biber,
  style=numeric,
  url=true,
  sorting=none
]{biblatex}


\renewcommand\multicitedelim{\addsemicolon\space}
\bibliography{$bibfile}

% Get copyright-symbol.
\usepackage{textcomp}

% Get shortmonthname.
\usepackage{datetime2}

% Get the [H] parameter for figures.
\usepackage{float}

% Get and set watermark.
\usepackage{eso-pic}
\AddToShipoutPictureFG{
  \begin{tikzpicture}[remember picture,overlay]
    \node[scale=15,rotate=45,opacity=0.1,color=red] at (current page.center) {Draft};
    \node[scale=3,rotate=45,opacity=0.1,color=red, yshift=-0.90cm] at (current
    page.center) {\DTMnow};
  \end{tikzpicture}
}

% Get heart.
\usepackage{txfonts}

% Get \isempty.
\usepackage{xifthen}

% Get strike through.
\usepackage[normalem]{ulem}

% Define t o d o commands.

\newcommand\tableofcontentsTODO{%
  \addcontentsline{toc}{section}{Todo-list}%
  \section*{\textbf{\textcolor{purple}{Todo-list:}}
      \@mkboth{\MakeUppercase\contentsname}{\MakeUppercaes\contentsname}%
    }%
    \vspace{-1.7cm}%
    \@starttoc{tocTODO}%
}%

\newcounter{counterTODO}\setcounter{counterTODO}{1}
\newcounter{counterTODOMaybe}\setcounter{counterTODOMaybe}{1}

\newcommand\TODO[2][]{%
  \ifshowTodo{%
    \def\varTODOtext{\textcolor{purple}{\texttt{<}\textbf{TODO}\#\arabic{counterTODO}\texttt{: #2>}}}
    \def\varTODOContentLine{\textcolor{purple}{#2}}
    \ifthenelse{\isempty{#1}}%
      {% Is not checked.
        \addcontentsline{tocTODO}{section}{\vspace{-0.3cm}\varTODOContentLine}%
        \varTODOtext\\%
      }%
      {% Is checked.
        \addcontentsline{tocTODO}{section}{\vspace{-0.3cm}\sout{\varTODOContentLine}}%
      }%
      \stepcounter{counterTODO}%
  }\fi%
}

\newcommand\TODOMaybe[1]{%
  \ifshowTodo{%
    \textcolor{orange}{%
      \texttt{<}\textbf{TODOMaybe}\#\arabic{counterTODOMaybe}\texttt{: #1?>}%
    }\\%
    \stepcounter{counterTODOMaybe}
  }\fi%
}
\newcommand\TODOCite{\ifshowTodo{\textcolor{blue}{\texttt{\textit{[Finalize
          citations]}}}}\else[?]\fi}
% Get adjustwidth command.
\usepackage{changepage}

% Get \NewEnviron command.
\usepackage{environ}


% Define notes environment.
\NewEnviron{notes}[1]
{
  \ifshowComments
    \vspace{-1.2cm}
    \color{teal}
    Notes (\textit{#1}):
    \begin{adjustwidth}{0.5cm}{0pt}
      \BODY
    \end{adjustwidth}
    \vspace{0.3cm}
  \fi
}


% Toggle clutter on or off.
\newif\ifshowTodo
\newif\ifshowComments

\ifdefined\visibleComments
  \showTodotrue
  \showCommentstrue
\else
  \showCommentsfalse
  \showTodofalse
\fi

\ifx\visibleComments\undefined
  \DeclareSourcemap{
    \maps[datatype=bibtex]{
      \map{
        \step[fieldsource=url, match=\regexp{http://ludwig.lub.lu.se/login}, final]
        \step[fieldset=url, null]
        \step[fieldset=urldate, null]
      }
    }
  }
\fi

\begin{document}

\pagenumbering{roman}

\newcommand\titleFirst{Using tailored web-interfaces to improve}
\newcommand\titleSecond{in-team communication and effectiveness}
\newcommand\titleThird{at MASSIVE, Malmö}
%\newcommand\titleFull{\titleFirst \ \titleSecond \ \titleThird}
\newcommand\titleFull{\titleFirst \ \titleSecond}

\newcommand\name{Stefan Eng}
\newcommand\email{kontakt@stefaneng.se}
\newcommand\nameEmail{\name \ \texttt{<\email>}}

\newcommand\DIVISION{ERGONOMICS AND AEROSOL TECHNOLOGY}
\newcommand\division{Ergonomics and Aerosol Technology}
\newcommand\supervisor{Johanna Persson}
\newcommand\contact{Patrick O'Casey}
\newcommand\examiner{Joakim Eriksson }
\newcommand\stanfordBootleg{\textit{design thinking bootleg}}

% Definition of names.
\def\shotgun{\texttt{Shotgun}}
\def\recordingSystem{\texttt{<Recording-System>}}
\def\flask{\texttt{Flask}}
\def\python{\texttt{Python}}
\def\reviewProtocol{\texttt{<Review-Protocol>}}
\def\numIterations{\texttt{<Num-Iterations>}}

% Begin title page.
{\color[rgb]{0.612,0.376,0.075}
  \thispagestyle{empty}

  \def\colorBorder{0.8cm}
  \def\titleBoxWidth{15.0cm}
  \definecolor{bgcolor}{rgb}{0.835,0.824,0.769}

  \begin{tikzpicture}[remember picture, overlay]
    \coordinate (topLeft) at ($%
      (current page.north west) + (\colorBorder, -\colorBorder)%
    $);

    \coordinate (bottom-Right) at ($%
      (current page.south east) + (-\colorBorder, \colorBorder)%
    $);

    \fill[fill=bgcolor] (topLeft) rectangle (bottom-Right);
  \end{tikzpicture}

  \begin{tikzpicture}[remember picture, overlay]
    \fill[fill=white] (0.5cm, 0cm) rectangle +(\titleBoxWidth+1cm, -8.0cm);
  \end{tikzpicture}

  \fontfamily{phv}\selectfont
  \vspace{1cm}
  %\vspace{1cm}
  \begin{minipage}{\paperwidth}
    \begin{minipage}{1.0cm}
      \
    \end{minipage}
    \begin{minipage}{\titleBoxWidth}
      \textrm{\YUGE \bfseries
        \titleFirst
        \vspace{0.3cm} \\
        \titleSecond
        %\vspace{0.45cm} \\
        %\titleThird
      } \\ \vspace{3mm}
      \hspace{-2mm}\noindent\rule{{\titleBoxWidth-2mm}}{0.3mm} \\
      \vspace{-5mm} \\
      \textrm{\LARGE
        \name
      }\vspace{5mm} \\
      {\footnotesize\bfseries
        DIVISION OF {\DIVISION} | DEPARTMENT OF DESIGN SCIENCES \\
        FACULTY OF ENGINEERING LTH | LUND UNIVERSITY \\
        \the\year
        \vspace{0.6cm} \\
        MASTER THESIS
      }
    \end{minipage}
  \end{minipage}

  \begin{tikzpicture}[remember picture, overlay]
    \node[anchor=center] at ($ (current page.south east) + (-4.2cm, 4.2cm) $) {
        \includegraphics[width=9cm]{LU-logotyp-tryck-digitalt/EPS_for_tryck/Engelska/LundUniversity_C2line_CMYK_cut.pdf}
    };
  \end{tikzpicture}

  \begin{tikzpicture}[remember picture, overlay]
    \node[anchor=center] at ($ (current page.south west) + ( 6.3cm, 4.2cm) $) {
        \includegraphics[width=9cm]{logos/Massive_Logotype_Print/Black/MASSIVE_LOGO_CMYK_BLACK.eps}
    };
  \end{tikzpicture}
}


%% Beginning Title page.
\newpage

\begin{center}
  {\HOne \titleFull} \\
  \vspace{2.5cm}
  {\Large \name} \\
  \vspace{2.0cm}
%  \hspace{0.17cm}
  \includegraphics[width=3.5cm]{LU-logotyp-tryck-digitalt/EPS_for_tryck/Engelska/LundUniversity_C2line_BLACK.eps}

\end{center}

%% End Title page.


%% Beginning Copyright page.
\newpage

%{\Large \titleFirst \\ \titleSecond}
{\Large \titleFull}
\vspace{1.5cm} \\
\begin{large}
  Copyright \textcopyright \ \the\year \ \name \\

  \textit{Published by} \vspace{0.2cm} \\
      Department of Design Sciences \\
      Faculty of Engineering LTH, Lund University \\
      P.O. Box 118, SE-221 00 Lund, Sweden

  \vspace{0.8cm}
  Subject: (MAMM10), Interaction Design \\
  Division: \division \\
  Supervisor: \supervisor \\
  \TODO[]{Co-supervisor or Company contact?}
  Co-supervisor / Company contact: \contact \\
  Examiner: \examiner \\


\end{large}
%% EndCopyright page.

\newpage


\section*{Abstract}

\TODO[]{Write non-mock abstract}
\def\abs{
  Working as a team in the gaming industry can be challenging as is, and it
  does not get better with the degradation of effectiveness and communication
  that potential stress of looming deadlines brings with it. \\

  This thesis examines if it is possible to counter the reduction
  in communication and effectiveness by making it easier to work with the
  available tooling. More specifically, it tries to answer the question; is is
  possible to reduce the negative effects of stress through tailor-made
  web-interfaces interfacing with available tooling that utilise findings from
  Cognitive Science and Interface Design?  \\

  The main method used in the study is design thinking, which is composed the
  following steps: empathize, define, ideate, prototype and test. During the
  empathize step interviews and surveys are conducted in order to get a measure
  of the how the users currently view working with {\shotgun}.
  Using the initial data, personas representing the requirements and needs of
  identified user-groups and stakeholders are created.
  After the define and ideate stages, manipulable prototypes are cerated using
  {\python} and {\flask}. The prototypes are then tested and monitored using
  {\recordingSystem} together with {\reviewProtocol} with the results
  feeding back into the prototype iteration cycle for a total of {\numIterations}
  iterations. \\

  By analyzing the observation data gathered using {\recordingSystem} the
  conclusion that {\texttt{<insert conclusion here>}} is (not) possible through
  usage of them method described above. \\

  \textbf{Keywords:}
    Interaction Design,
    Tooling,
    Stress,
    Development,
    Design-Thinking.
}
{\abs}


\newpage

\section*{Sammanfattning}

\TODO[]{Complete English version \& translate to Swedish}
{\abs}

\newpage

\section*{Preface}

\TODO[]{Finish preface}

{\large

First off I want to thank my fiancé, Amanda, for helping out with proof-reading and
edit-suggestions while simultaneously putting up with the weird question and
theories lobbed in her general direction.
\textcolor{red}{\ensuremath\varheartsuit} \\

\vspace{0.7cm}
\DTMsavenow{thisone}
Lund,
\DTMenglishmonthname{\DTMfetchmonth{thisone}},
\DTMfetchyear{thisone}
\vspace{0.7cm} \\
\name
}

\newpage

\tableofcontents

\newpage

\pagenumbering{arabic}

\section{Introduction}

  \begin{notes}{Introduction structure | CARS}
    Create A Research Space \\
    Type: General $\rightarrow$ Specific
    \begin{itemize}
      \item{Establish your territory}
      \begin{itemize}
        \item{Make a claim}
        \item{Review previous work}
        \item{Position yourself}
      \end{itemize}
      \item{Establish your own niche}
        \begin{itemize}
          \item{Raise a problem}
          \item{Indicate a gap}
          \item{Continue}
        \end{itemize}
      \item{Occupy your space}
        \begin{itemize}
          \item{Present a statement of purpose}
          \item{Outline / Blueprint / Mapping}
          \item{Broader significance}
        \end{itemize}
    \end{itemize}
  \end{notes}

  \TODO[]{Finish Introduction}

  Communication and effectiveness have always been valued in industry,
  especially now in the modern age {\TODOCite}. \\


  \subsection{Issues and aim}

    \TODO[]{Finish Issues and aim}

  \subsection{Scope and limitations}

    \TODO[]{Finish Scope and limitation}

  \subsection{Report structure}

    \TODO[]{Finish Report structure}

\section{Methodology}

  \begin{notes}{Methodology structure | STAR}
    Type: Specific \\
    Sufficiency, Typicality, Accuracy, Relevance.
    \begin{itemize}
      \item{Convince readers that materials are valid}
        \begin{itemize}
          \item{follows agreed protocol}
          \item{What?}
          \item{How?}
          \item{What Materials?}
          \item{Problems? / How solved?}
        \end{itemize}
      \item{Materials should}
      \begin{itemize}
        \item{support the claim}
        \item{be typical \& representative}
        \item{be accurate \& up-to-date}
        \item{be relevant to the claims made}
      \end{itemize}
    \end{itemize}
  \end{notes}

  \TODO[]{Finish Methodology}

  \subsection{Design thinking}

    After coming in
    contact with the method in a paper on using personas to evaluate user
    requirements\cite{c_personas_article_controlled_experiment} design thinking
    became an interesting method candidate due to the following characteristics.
    As a process, design thinking%
      \cite{c_design_thinking_article_origins,
      c_design_thinking_online_nonpeer_dschool_stanford},
    has a strong focus on user requirements and
    promotes an iteration heavy approach to prototyping and evaluation. Since
    both of these characteristics were brought up in the initial discussions of
    what kind of process that should underpin this project, it was chosen as the
    projects main method. The process is generally defined by five steps or
    \textit{modes}; empathize, define, ideate, prototype and test, each
    described in more detail below.

    %After coming in contact with the design thinking process through a paper on
    %using personas for evaluation of
    %users requirements\cite{c_personas_article_controlled_experiment}, and
    %further reading about it\cite{c_design_thinking_article_origins}%
    %\cite{c_design_thinking_online_nonpeer_dschool_stanford}
    %it was considered a good fit for this project due to its user centered
    %approach and focus on a iterative process. The main structure of the
    %design thinking consists of five steps; empathize, define, ideate prototype
    %and test, each defined below.

    \subsubsection{Empathize}

      This step expands collecting end-user requirement and paint points
      to also include the requirements and frustrations of related stakeholders
      (henceforth collectively referred to as users).
      Using this extended pool of users, the goal is to observe, engage and
      immerse one self in the relationship between the users and the current
      situation. Observing consists of noting how users interact with their
      environment; capture quotes, watch common usages and be on the lookout for
      clues that give hints about their feelings and thoughts. Try to reveal
      deeper insights by interacting with users more directly through
      interviews and similar means. Be immersed by constructing or participating
      in the specific environments that could influence the design or
      perceptions about the situation. \\

      As the name suggests, the empathize stage is about probing the emotions
      that drive the users needs and behaviours, in order to make it possible to
      design innovate solutions to their problems.

      %gathering the insights
      %necessary to design innovative solutions by probing the emotions that
      %drive the users needs and behaviours.

      %Design thinking expands on the usual user-requirement gathering with the
      %inclusion of formulating and gathering requirements for relevant
      %stakeholders.

    \subsubsection{Define}

      Your point of view; by taking the insights gathered in the previous step,
      construct an actionable problem statement that reflects the understanding
      of the users and their environment, this is your point of view.
      Mainly, this should be more than a simple re-statement of the problem
      space, it should be a unique design solution strongly influenced and
      framed by the discovered needs of the end-users. It is important that any
      initial or revised problem statement is continuously re-framed to mach new
      insights generated from current and future empathy work.

    \subsubsection{Ideate}

      In this mode the goal is to do a wide examination of the solution space by
      continuing to generate ideas well beyond any initial solutions that spring
      to mind. After such probing is done, the focus is narrowed and the
      solutions are evaluated one by one in order to determine if they are
      viable or not. \\

        \TODO[]{Create ideate-focus-cycle-graphic.pdf}

      It is important to insist that the different focus phases should be fully
      separated from each other. There should be no evaluation in the broad
      phase, only collection, similarly, introducing new ideas during the
      evaluation phase should be avoided.

      \TODO[]{Include break intrinsic bias / framing á design}

    \subsubsection{Prototype}

      Prototyping should manifests the current ideas into the physical world in
      such a way that they can be interacted with and the results discussed. An
      initial prototype should be cheap and rough; a role-playing activity,
      post-its on a wall or a simple physical object. By keeping the investment
      in, and resolution of the prototype low it will be easier to quickly explore
      different scenarios and incorporate the results back into the iteration
      process. \\

      Prototyping will most readily generate data that evaluate the pure
      functionality of the idea that the prototype is based on, but it also has
      several other benefits. Interacting with the prototype can add to the
      empathy work by deepening the understanding of the users and the
      surrounding design space.  It is also possible to run tests on the
      prototype artefact in order to validate or refine current theories and
      solutions.  Further, the use of a prototype can facilitate easier
      communicating of the underlying design vision for both showcase and
      inspiration. Finally, by using multiple different prototypes at the same
      time, parallell execution and exploration of multiple ideas can be
      preformed at once, increasing overall throughput.

    \subsubsection{Test}

    As described in the Standford d.school's {\stanfordBootleg}%
    \cite[V]{c_design_thinking_online_nonpeer_dschool_stanford};
      \enquote{Prototype as if your know you're right, but test as if you know
        you're wrong}. This phase allows you to gather feedback, refine
      solutions and further deepen the empathy work. The core of the process is
      an iterative sequence where data, gathered by observing users  interacting
      with a prototype in the appropriate context, is used to further refine the
      design- and problem-space.

  \subsection{Design thinking and this \{...\} (project?)}

    \TODO[]{Figure the correct word for '<this current document>'}
    \TODO[]{How is design thinking used in this <project?{>}}

  \subsection{Theories}
    \TODO[]{Finish Theories}
  \subsection{Tools}
    \TODO[]{Finish Tools}
  \subsection{Threats to validity}
    \TODO[]{Finish Threats to validity}

\section{Results}
  \begin{notes}{Results}
    Type: Specific
    \begin{itemize}
      \item{How should the result be presented?}
      \item{Present part or all of it?}
      \item{Include visualisations}
      \item{Don't retell findings that are directly represented by the visuals}
      \item{Decide which results should be brought to the foreground}
    \end{itemize}
  \end{notes}

  \TODO[]{Finish Results }
  \TODOMaybe{Flesh out more sub-sections}

\section{Discussion and conclusion}

  \begin{notes}{Discussion}
    Type: Specific $\rightarrow$ General
    \begin{enumerate}
      \item{Present relevant background information}
      \item{Summarize key results}
      \item{Discuss the results}
      \item{Where there any limitations? How were they solved?}
      \item{Suggest broader implications \& further research}
    \end{enumerate}
    \begin{itemize}
      \item{Discuss \& interpret your findings}
      \item{Restate results $\rightarrow$ broaden}
    \end{itemize}
  \end{notes}

  \TODO[]{Finish Discussion and conclusion}

\newpage
\section{References}

  \TODO[]{Finish References}
  \TODO[]{Add retrived-on dates to relevant references}

  \printbibliography[heading=none]

\section{Appendices}


\section{Popular science summary}

  \TODO[]{Finish Popular science summary}

  \begin{figure}[H]
    \begin{center}
      \includegraphics{figures/plots/fig_test.pdf}
      \caption{Hello line figure!}
    \end{center}
  \end{figure}

  \begin{figure}[H]
    \begin{center}
      \vspace{-0.3cm}
      \includegraphics{figures/plots/fig_pie_test.pdf}
      \vspace{-0.5cm}
      \caption{Hello pie figure!}
    \end{center}
  \end{figure}

  \begin{figure}[H]
    \begin{center}
      \includegraphics{figures/plots/fig_bar_test.pdf}
      \caption{Hello bar figure!}
    \end{center}
  \end{figure}

\ifdefined\visibleComments
  \newpage
  \tableofcontentsTODO
\fi

\end{document}

\documentclass{article}

\usepackage[utf8]{inputenc}
\usepackage[english,swedish]{babel}
\usepackage[T1]{fontenc}

\setlength{\parindent}{0cm}

\begin{document}

\begin{center}

  % Arbetstitel.
  Examensarbete -- Måldokument. \\
  \ \\
  {\large
    \textbf{För mycket av det goda} -
    Att använda interaktionsdesign och kognitionsvetenskap för
    att minska friktion i informationsflödet hos Massive
    Entertainment, Malmö. \\
  }
  %} \\
  %--- \\
  \ \\
  Prel. period: 2018-09-17 - 2019-02-18\\
  Stefan Eng \texttt{<atn08sen@student.lu.se>}
  \ \\
  ---
  \vspace{-0.3cm}

\end{center}

\section*{Bakgrund}

  Efter kontakt och diskussion med personal på Massive så har det framgått att
  det överenskomna arbetes- och kommunikations-flödet inte fungerat som
  planerat mellan de olika disciplinerna i ett projekt vid ett tillfäll då
  extra press har uppstått.

\section*{Motiv}

  Även fast det tänkta målet blev uppnått så framgår det tydligt att det hade
  varit önskvärt och mindre ansträngande om kommunikationen via de
  tilltänkta systemen hade fortsatt även då en pressad situation uppstår.

\section*{Mål \& Problemställning}

  Målet är att använda litteratur från kognitionsvetenskap och interaktionsdesign
  för att identifiera och söka lösningar till eventuell friktion inom
  planerings- och kommunikations-systemen som används internt. \\

  Problemställningen handlar om huruvida det går att producera en
  digital prototyp, grundad på lösningar från litteraturen, som mätbart
  reducerar den upplevda friktionen och höjer ribban för mängden press som kan
  uppstå innan användare överger systemet.

\section*{Metodik}

  Den preliminära metodiken enligt följande:
  \begin{enumerate}
    \item{Intervjua personal och identifiera upplevda problem.}
    \item{Använda litteraturen för att verifiera de mest kritiska problemen och
        hitta eventuella existerande lösningar.}
    \item{Välja en specifik lösning till ett upplevt problem och konstruera en
        prototyp enligt lösningsförslag.}
    \item{Testa prototypen mot personal och mäta huruvida lösningen reducerar
        det upplevda problemet eller inte.}
  \end{enumerate}

  Metoden kommer följa ett cykliskt arbetsflöde där ett flertal problem samlas
  in i punkt 1 och sedan reduceras i mängd med avseende på upplevd negativ
  effekt alternativt, teoretisk relevans i punkt 2. I punkt 3 kommer sedan en
  lösning till en av de utsorterade problemen att implementeras, för att sedan
  testa effekten av implementeringen i punkt 4. \\

  Cykeln 3 $\rightarrow$ 4 repeteras så länge det finns tid och kvarvarande
  lösningsförslag. Om lösningsförslagen tar slut; återvänd till punkt 2 och
  försök identifiera lösningar till de problem som sorterats ut och
  implementera dessa. Tar även dessa slut, återgå till punkt 1 för att
  identifiera nya problem bland användarna.

\section*{Vetenskaplig grund}

  \textbf{The design of everyday things}\cite{c_design_norman} -
  Hur kognitionsvetenskap kan användas i design och underlätta användning. \\

  \textbf{Interaction design : beyond human-computer
    interaction}\cite{c_interaction_design} -
  Hur fungerar interaktionsdesign, förstå sig på användaren, hur gränssnitt
  påverkar, behovsidentifiering m.m.

\section*{Bidragande kunskapsutveckling}

  Detta arbetet ger möjlighet att testa kognitionsvetenskap- och
  interaktionsdesign-teori och dess praktiska tillämpningar i spelbranchen,
  en krävande interaktions-miljö där problemfri kommunikation av information och
  hur system används på bästa sätt är av yttersta vikt.

\section*{Resurser}

  För att utföra arbetet behövs preliminärt;
  \begin{enumerate}
    \item{Tillgång till personal på Massive Entertainment för intervjuer.}
    \item{Dator med möjlighet att sammanställning data och utveckla
        interaktions-prototyp(er).}
    \item{Tillgång till relevanta interna kommunikations- och planerings-system
        på Massive för analys och insamling av information.}
    \item{Stöd med information och vägledning kring interna processer och
        metoder inom Massive.}
    \item{Information och stöd med insamling och bearbetning av teori rörande
        interaktionsdesign.}
    \item{Delvist stöd med vägledning för information kring Kognition.}
  \end{enumerate}

  Punkt 1-4 löses genom fysisk arbetsplats på Massive Entertainment i Malmö,
  samt företags-kontakt på plats; Patrick O'Casey (Cinematics Producer) \\
  \texttt{<patrick.ocasey@massive.se>}. \\

  Punkt 5 löses via kontakt med främst handledare; <namn-handledare> \\
  \texttt{<email@handledare>} och vidare examinator; <namn-examinator> \\
  \texttt{<email@examinator>} från Institutionen för arkitektur och byggd
  miljö, LTH. \\

  Punkt 6 löses via kontakt med Annika Wallin (Docent Kognitionsvetenskap,
  Filosofiska intuitionen, Lunds Universitet)
  \texttt{<annika.wallin@lucs.lu.se>} som ställt upp för frågor och förslag
  kring relevant litteratur inom Kognitionsvetenskap om behovet uppstår.

\bibliography{mdoc}
\bibliographystyle{plain}

\end{document}

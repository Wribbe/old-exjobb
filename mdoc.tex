\documentclass{article}

\usepackage[utf8]{inputenc}
\usepackage[english,swedish]{babel}
%\usepackage{datetime2}

\setlength{\parindent}{0cm}

\begin{document}

\begin{center}

  % Arbetstitel.
  Examensarbete -- Måldokument. \\
  \ \\
  {\large
    \textbf{För mycket av det goda} -
    Att använda interaktionsdesign och kognitionsteori för
    att minska friktion kring projektinformation hos Massive
    Entertainment, Malmö. \\
  }
  %} \\
  %--- \\
  \ \\
  Prel. period: 2018-09-17 - 2019-02-18\\
  Stefan Eng \texttt{<atn08sen@student.lu.se>}
  \ \\
  ---
  \vspace{-0.3cm}

\end{center}

\section*{Bakgrund}

  Efter kontakt och diskussion med personal på Massive så har det framgått att
  det överenskomna arbetes- och kommunikations-flödet inte fungerat som
  planerat mellan de olika disciplinerna i ett projekt vid tillfällen då extra
  press har uppstått.

\section*{Motiv}

  Även fast det tänkta målet blev uppnått så framgår det tydligt att det hade
  varit önskvärt och mindre ansträngande om kommunikationen via de
  tilltänkta systemen hade fortsatt även då en pressad situation uppstår.

\section*{Mål \& Problemställning}

  Målet är att använda litteratur från kognitionsteori och interaktionsdesign
  för att identifiera och söka lösningar till eventuell friktion inom
  planerings- och kommunikations-systemen som används internt. \\

  Problemställningen handlar om huruvida det går att producera en
  digital prototyp, grundad på lösningar från litteraturen, som mätbart
  reducerar den upplevda friktionen och höjer ribban för mängden press som kan
  uppstå innan användare överger systemet.

\section*{Vetenskaplig grund (ref. artiklar etc.)}

\section*{Bidragande kunskapsutveckling}

\section*{Resurser}

\end{document}

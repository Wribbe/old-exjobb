\documentclass{article}

\usepackage[utf8]{inputenc}
\usepackage[english,swedish]{babel}
\usepackage[T1]{fontenc}
\usepackage{pgfgantt}
\usepackage{pdflscape}

\setlength{\parindent}{0cm}

\begin{document}

\begin{center}

%Projektplanering Alla större arbeten tjänar på att planeras, inte minst ett
%examensarbete som pågår ungefär ett halvår. Det kan vara en god idé att använda
%det tidigare skrivna måldokumentet som en stomme till en projektplan. I
%projektplanen beskrivs arbetets innehåll, avgränsningar och metodik relativt
%detaljerat. Detta innebär bland annat att försöka planera vilka mätningar,
%beräkningar, intervjuer eller andra undersökningar som ska utföras, och hur man
%avser att tolka och analyser de resultat som detta ger. En sådan plan kommer
%givetvis att variera högst väsentligt mellan olika examensarbeten.
%
%En annan viktig del i planeringen är upprättande av en tidsplan. För att
%underlätta styrning och uppföljning av projektet bör arbetet delas upp i mindre
%delmoment och avstämningspunkter. Det är lätt hänt att tiden rinner iväg,
%vilket i viss mån kan regleras med en tidsplan. En tidsplan ska, så bra som
%möjligt, uppskatta den tid som de olika momenten i examensarbetet kan ta. Det
%är viktigt att på ett tidigt stadium av arbetet få en uppfattning av hur mycket
%tid som finns tillgänglig för de olika delmoment som ingår. Ett examensarbete
%ska motsvara 20 veckors heltidsstudier. Begränsningen är givetvis inte strikt,
%men kan vara ett riktmärke att jobba efter. Det är viktigt att även ta hänsyn
%till konkurrerande aktiviteter såsom studentens/studentgruppens eventuella
%parallella studier och handledarens eventuella längre tjänsteresor.
%
%Det kan också vara bra att fundera över formerna för handledning på ett tidigt
%stadium, och klargöra respektive parters förväntningar på varandra. En del
%tycker att en form med regelbundna möten fungerar bra, andra trivs med att
%träffas efter behov. En projektplan ska definiera avstämningspunkter med alla
%handledare.
%
%Tänk på att inte bara planera för själva utredningsarbetet utan också ta med
%aktiviteter i slutfasen t.ex. opponering och presentation där diverse praktiska
%saker som tid för tryckning av rapporten, presentationsförberedelser och
%liknande kan påverka tidsplanen för arbetet.
%
%Planering inte är något statiskt. Många examensarbeten har karaktär av
%forskning eller utveckling, vilket innebär att man inte vet riktigt vad man har
%att vänta sig. En god planering måste vara tillräckligt flexibel för att ta
%hänsyn till detta. Projektplanen bör uppdateras löpande och förändringar bör
%diskuteras med huvudhandledaren.

  % Arbetstitel.
  Examensarbete -- Måldokument. \\
  \ \\
  {\large
    \textbf{För mycket av det goda} -
    Att använda interaktionsdesign och kognitionsvetenskap för
    att minska friktionen i informationsflödet hos Massive
    Entertainment, Malmö. \\
  }
  %} \\
  %--- \\
  \ \\
  Prel. period: 2018-09-17 - 2019-02-18\\
  Stefan Eng \texttt{<atn08sen@student.lu.se>}
  \ \\
  ---
  \vspace{-0.3cm}

\end{center}

\section*{Bakgrund}

  Vid diskussion med personal på Massive kom det upp att det digitala
  arbetes- och kommunikations-flödet mellan olika discipliner i ett
  projekt övergetts för fysisk kommunikation via boards och post-it's  vid ett
  tillfälle av extra press.

\section*{Motiv}

  Även fast det tänkta målet blev nått så framgår det tydligt från de
  tillfrågade att det hade varit önskvärt, och mindre ansträngande om
  kommunikationen via de tilltänkta systemen hade fortsatt, trots press.

\section*{Mål \& Problemställning}

  Målet är att använda litteratur från kognitionsvetenskap och interaktionsdesign
  för att identifiera och söka lösningar till eventuell friktion inom
  planerings- och kommunikations-systemen som används internt. \\

  Problemställningen handlar om huruvida det går att producera en
  digital prototyp, grundad på lösningar från litteraturen, som mätbart
  reducerar den upplevda friktionen. Tanken är att införa modifikationer som
  höjer ribban för mängden press som kan uppstå innan användare väljer att
  överge det digitala systemet.

\section*{Metodik}

  Den preliminära metodiken enligt följande:
  \begin{enumerate}
    \item{Intervjua personal och identifiera upplevda problem.}
    \item{Använda litteraturen för att verifiera de mest kritiska problemen och
        formulera lösningar till dessa eller hitta eventuella existerande
        lösningar.}
    \item{Välja en specifik lösning till ett upplevt problem och konstruera en
        prototyp i syfte att minimera, alternativt eliminera det upplevda
        problemet.}
    \item{Testa prototypen mot personal och mäta huruvida lösningen påverkar
        det upplevda problemet eller inte.}
  \end{enumerate}

  Metoden kommer följa ett cykliskt arbetsflöde där ett flertal problem samlas
  in i punkt 1 och sedan reduceras i mängd med avseende på upplevd negativ
  effekt alternativt, teoretisk relevans i punkt 2. I punkt 3 väljs sedan en
  lösning till en av de utsorterade problemen och implementeras, för att sedan
  testa effekten av implementeringen i punkt 4. \\

  Cykeln 3 $\rightarrow$ 4 repeteras så länge det finns tid och kvarvarande
  lösningsförslag. Om lösningsförslagen tar slut; återvänd till punkt 2 och
  försök identifiera lösningar till de problem som sorterats ut och
  implementera dessa. Tar även dessa slut, återgå till punkt 1 för att
  identifiera nya problem bland användarna.

\section*{Vetenskaplig grund}

  \textbf{The design of everyday things}\cite{c_design_norman} -
  Hur kognitionsvetenskap kan användas i design och underlätta användning. \\

  \textbf{Interaction design : beyond human-computer
    interaction}\cite{c_interaction_design} -
  Hur fungerar interaktionsdesign, förstå sig på användaren, hur gränssnitt
  påverkar, behovsidentifiering m.m.

\section*{Bidragande kunskapsutveckling}

  Detta arbetet ger möjlighet att praktiskt testa teorier från
  kognitionsvetenskap och interaktionsdesign och dess tillämpningar i
  spelbranchen, en krävande interaktions-miljö där problemfri kommunikation av
  information och hur system används på bästa sätt är av yttersta vikt.

\section*{Resurser}

  För att utföra arbetet behövs preliminärt;
  \begin{enumerate}
    \item{Tillgång till personal på Massive Entertainment för intervjuer.}
    \item{Dator med möjlighet att sammanställning data och utveckla
        interaktions-prototyp(er).}
    \item{Tillgång till relevanta interna kommunikations- och planerings-system
        på Massive för analys och insamling av information.}
    \item{Stöd med information och vägledning kring interna processer och
        metoder inom Massive.}
    \item{Information och stöd med insamling och bearbetning av teori rörande
        interaktionsdesign.}
    \item{Delvist stöd med vägledning för information kring
        kognitionsvetenskap.}
  \end{enumerate}

  Punkt 1-4 löses genom fysisk arbetsplats på Massive Entertainment i Malmö,
  samt företags-kontakt på plats; Patrick O'Casey (Cinematics Producer) \\
  \texttt{<patrick.ocasey@massive.se>}. \\

  Punkt 5 löses via kontakt med främst handledare; \\
  Johanna Person (Biträdande universitetslektor vid Ergonomi och
  aerosolteknologi, LTH) \texttt{<johanna.persson@design.lth.se>} och vidare
  examinator;
  \\Joakim Eriksson (Forskningsingenjör vid Ergonomi och aerosolteknologi, LTH)
  \texttt{<joakim.eriksson@design.lth.se>}. \\

  Punkt 6 löses via kontakt med Annika Wallin (Docent Kognitionsvetenskap,
  Filosofiska intuitionen, Lunds Universitet)
  \texttt{<annika.wallin@lucs.lu.se>} som ställt upp för frågor och förslag
  kring relevant litteratur inom kognitionsvetenskap om behovet uppstår.

\newpage

\begin{landscape}

  \center
  \section*{Tidsschema}

  \input{tex/tidsschema.tex}

\end{landscape}

\end{document}

\documentclass{article}

\usepackage[utf8]{inputenc}
\usepackage[english,swedish]{babel}
\usepackage[T1]{fontenc}
\usepackage{wasysym}
\usepackage{pgfgantt}
\usepackage{pdflscape}
\usepackage{geometry}
\usepackage{url}
\usepackage{hyperref}
\usepackage{xifthen}
\usepackage{setspace}
\usepackage{pgffor}

\setlength{\parindent}{0cm}

\newcounter{countQuestions}
\newenvironment{question}
  {%
%    \begin{doublespacing}
  }%
  {%
%    \end{doublespacing}
  }%

\newcommand{\qmline}[2][]{%
  #2? \\ \ \\
  \ifthenelse{\isempty{#1}}%
    {\rule{\textwidth}{0.4pt} \\ \\ \rule{\textwidth}{0.4pt} \\ }%first is empty.
    {%
      \foreach \index in {1, ..., #1} {%
        \rule{\textwidth}{0.4pt} \\ \\
      }
    }%first is not empty.
    \\
}

\newcommand{\qline}[2][]{%
  \ifthenelse{\isempty{#1}}%
    {#2: }%first is empty.
    {#2 [\textit{Ex: #1}]: }%first is not empty.
    \hrulefill \\ \\
}
\newcommand{\hh}{\hspace{1cm}}
\newcommand{\qscale}[1]{%
  \textit{#1}: \\
    Strongly Disagree\ 1---2---3---4---5---6---7\ Strongly Agree
  %  \fullmoon \ Never  \hh
  %  \fullmoon \ Sometimes  \hh
  %  \fullmoon \ Most of the time  \hh
  %  \fullmoon \ Always
  \ \\ \\
}


\begin{document}

\begin{center}

%Projektplanering Alla större arbeten tjänar på att planeras, inte minst ett
%examensarbete som pågår ungefär ett halvår. Det kan vara en god idé att använda
%det tidigare skrivna måldokumentet som en stomme till en projektplan. I
%projektplanen beskrivs arbetets innehåll, avgränsningar och metodik relativt
%detaljerat. Detta innebär bland annat att försöka planera vilka mätningar,
%beräkningar, intervjuer eller andra undersökningar som ska utföras, och hur man
%avser att tolka och analyser de resultat som detta ger. En sådan plan kommer
%givetvis att variera högst väsentligt mellan olika examensarbeten.
%
%En annan viktig del i planeringen är upprättande av en tidsplan. För att
%underlätta styrning och uppföljning av projektet bör arbetet delas upp i mindre
%delmoment och avstämningspunkter. Det är lätt hänt att tiden rinner iväg,
%vilket i viss mån kan regleras med en tidsplan. En tidsplan ska, så bra som
%möjligt, uppskatta den tid som de olika momenten i examensarbetet kan ta. Det
%är viktigt att på ett tidigt stadium av arbetet få en uppfattning av hur mycket
%tid som finns tillgänglig för de olika delmoment som ingår. Ett examensarbete
%ska motsvara 20 veckors heltidsstudier. Begränsningen är givetvis inte strikt,
%men kan vara ett riktmärke att jobba efter. Det är viktigt att även ta hänsyn
%till konkurrerande aktiviteter såsom studentens/studentgruppens eventuella
%parallella studier och handledarens eventuella längre tjänsteresor.
%
%Det kan också vara bra att fundera över formerna för handledning på ett tidigt
%stadium, och klargöra respektive parters förväntningar på varandra. En del
%tycker att en form med regelbundna möten fungerar bra, andra trivs med att
%träffas efter behov. En projektplan ska definiera avstämningspunkter med alla
%handledare.
%
%Tänk på att inte bara planera för själva utredningsarbetet utan också ta med
%aktiviteter i slutfasen t.ex. opponering och presentation där diverse praktiska
%saker som tid för tryckning av rapporten, presentationsförberedelser och
%liknande kan påverka tidsplanen för arbetet.
%
%Planering inte är något statiskt. Många examensarbeten har karaktär av
%forskning eller utveckling, vilket innebär att man inte vet riktigt vad man har
%att vänta sig. En god planering måste vara tillräckligt flexibel för att ta
%hänsyn till detta. Projektplanen bör uppdateras löpande och förändringar bör
%diskuteras med huvudhandledaren.

  % Arbetstitel.
  Examensarbete. \\
  \ \\
  {\large
    \textbf{Projektplan för examensarbete}
  } \\
  %} \\
  %--- \\
  \ \\
  Stefan Eng \texttt{<atn08sen@student.lu.se>}
  \ \\
  \today \\
  \ \\
  ---
  \vspace{-0.3cm}

\end{center}

\section*{Innehåll}
  Arbetet kommer att undersöka i vilken utsträckning forsknings-resultat
  från, samt teorier inom området människa-datorinteraktion, MDI\cite{c_mdi},
  eller engelska motsvarigheten, HDI\cite{c_hci} kan underlätta användningen av
  Shotgun\cite{c_shotgun} på Massive Entertainment\cite{c_massive}, Malmö.
  Målet är främst att mäta om en skräddarsydd interaktionslösning har en
  positiv inverkan på användareupplevelsen, och sekundärt, om den är stor nog
  att motivera det initial arbetet.

\section*{Avgränsningar}

  Arbetet kommer främst fokusera på att analysera interaktioner med
  Shotgun\cite{c_shotgun}, en mjukvaran som används för att skapa och hålla
  uppsikt på utvecklingen av multimedia projekt. Vid diskussion har det framkommit
  att det finns ytterligare system som hade varit intressanta att analysera,
  men som ej kommer utföras på grund av begränsad tid.

  \subsection*{Arbetsflöden}

    Målet är att identifiera mellan tre till tio ofta förekommande arbetsflöden
    som består av minst tre distinkta handlingar som exempelvis fil-operationer
    och mus-klick där användare upplevt irritation eller att det kan ske
    förbättringar.

  \subsection*{Personal och Tid}

    Vid tre utförda tester rekommenderas minst fyra till fem personer i varje
    grupp för att upptäcka majoriteten av de problem som
    existerar\cite[p.126]{c_handbook_usability}. Givet fem personer i varje
    grupp, kommer det behövas femton personer för att slutföra alla tester.
    Då tid är en bristvara är målet att tillbringa runt 60 minuter med varje
    person och testtillfälle, med en övre gräns på 90 minuter.

\section*{Metodik}

  Övergripande kommer studien

Thinking Aloud \cite[p.204]{c_handbook_usability}
Simple Single-Room Setup \cite[p.101]{c_handbook_usability}

  Inledningsvis kommer tre till fyra personer att väljas ut till en
  pilotstudie för att säkerställa en korrekt riktning på testuppställningen och
  frågor.

  För att etablera ett startvärde att jämföra mot samt få en överblick, kommer
  inledande intervjuer och enkäter om hur den nuvarande processen upplevs
  utföras bland utvecklare och chefer. \\

  [ ... ]



  %Inledningsvis behövs ett värde som den slutliga analysen kan utvärderas mot

  %Utvärdering av

  %Inledningsvis kommer intervjuer ske med de anställda för att identifiera
  %upplevda problem-områden samt skaffa sig ett initialvärde för att

\newpage
\section*{Mätningar}

\begin{question}
  \qline{Alias}
  Age: \hspace{0.5cm}
    \fullmoon \ $<30$ \hh
    \fullmoon \ $31-40$ \hh
    \fullmoon \ $41-50$ \hh
    \fullmoon \ $51+$
    \ \\ \\
  \qline[Gameplay Animator]{Position}
  \qscale{I am often annoyed by the software}
  \qscale{I am often annoyed by quizzes}
  \qscale{I always know where in the process my colleges are}
  \qmline{What problems do you experience with the software}
  \qmline[4]{What positive aspects do you experience with the software}
\end{question}

\newgeometry{left=3cm, bottom=3cm}
\begin{landscape}
  \section*{Tidsschema}
  \centering
  \input{tex/tidsschema.tex}
\end{landscape}
\restoregeometry

\newpage

\bibliography{pplan}
\bibliographystyle{unsrt}

\end{document}

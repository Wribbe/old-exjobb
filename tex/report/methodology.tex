\subsection{Design thinking}

  After coming in
  contact with the method in a paper on using personas to evaluate user
  requirements\cite{c_personas_article_controlled_experiment} design thinking
  became an interesting method candidate due to the following characteristics.
  As a process, design thinking%
    \cite{c_design_thinking_article_origins,
    c_design_thinking_online_nonpeer_dschool_stanford},
  has a strong focus on user requirements and
  promotes an iteration heavy approach to prototyping and evaluation. Since
  both of these characteristics were brought up in the initial discussions of
  what kind of process that should underpin this project, it was chosen as the
  projects main method. The process is generally defined by five steps or
  \textit{modes}; empathize, define, ideate, prototype and test, each
  described in more detail below.

  %After coming in contact with the design thinking process through a paper on
  %using personas for evaluation of
  %users requirements\cite{c_personas_article_controlled_experiment}, and
  %further reading about it\cite{c_design_thinking_article_origins}%
  %\cite{c_design_thinking_online_nonpeer_dschool_stanford}
  %it was considered a good fit for this project due to its user centered
  %approach and focus on a iterative process. The main structure of the
  %design thinking consists of five steps; empathize, define, ideate prototype
  %and test, each defined below.

  \subsubsection{Empathize}

    This step expands collecting end-user requirement and paint points
    to also include the requirements and frustrations of related stakeholders
    (henceforth collectively referred to as users).
    Using this extended pool of users, the goal is to observe, engage and
    immerse one self in the relationship between the users and the current
    situation. Observing consists of noting how users interact with their
    environment; capture quotes, watch common usages and be on the lookout for
    clues that give hints about their feelings and thoughts. Try to reveal
    deeper insights by interacting with users more directly through
    interviews and similar means. Be immersed by constructing or participating
    in the specific environments that could influence the design or
    perceptions about the situation. \\

    As the name suggests, the empathize stage is about probing the emotions
    that drive the users needs and behaviours, in order to make it possible to
    design innovate solutions to their problems.

    %gathering the insights
    %necessary to design innovative solutions by probing the emotions that
    %drive the users needs and behaviours.

    %Design thinking expands on the usual user-requirement gathering with the
    %inclusion of formulating and gathering requirements for relevant
    %stakeholders.

  \subsubsection{Define}

    Your point of view; by taking the insights gathered in the previous step,
    construct an actionable problem statement that reflects the understanding
    of the users and their environment, this is your point of view.
    Mainly, this should be more than a simple re-statement of the problem
    space, it should be a unique design solution strongly influenced and
    framed by the discovered needs of the end-users. It is important that any
    initial or revised problem statement is continuously re-framed to mach new
    insights generated from current and future empathy work.

  \subsubsection{Ideate}

    In this mode the goal is to do a wide examination of the solution space by
    continuing to generate ideas well beyond any initial solutions that spring
    to mind. After such probing is done, the focus is narrowed and the
    solutions are evaluated one by one in order to determine if they are
    viable or not. \\

      \TODO[]{Create ideate-focus-cycle-graphic.pdf}

    It is important to insist that the different focus phases should be fully
    separated from each other. There should be no evaluation in the broad
    phase, only collection, similarly, introducing new ideas during the
    evaluation phase should be avoided.

    \TODO[]{Include break intrinsic bias / framing á design}

  \subsubsection{Prototype}

    Prototyping should manifests the current ideas into the physical world in
    such a way that they can be interacted with and the results discussed. An
    initial prototype should be cheap and rough; a role-playing activity,
    post-its on a wall or a simple physical object. By keeping the investment
    in, and resolution of the prototype low it will be easier to quickly explore
    different scenarios and incorporate the results back into the iteration
    process. \\

    Prototyping will most readily generate data that evaluate the pure
    functionality of the idea that the prototype is based on, but it also has
    several other benefits. Interacting with the prototype can add to the
    empathy work by deepening the understanding of the users and the
    surrounding design space.  It is also possible to run tests on the
    prototype artefact in order to validate or refine current theories and
    solutions.  Further, the use of a prototype can facilitate easier
    communicating of the underlying design vision for both showcase and
    inspiration. Finally, by using multiple different prototypes at the same
    time, parallell execution and exploration of multiple ideas can be
    preformed at once, increasing overall throughput.

  \subsubsection{Test}

  As described in the Standford d.school's {\stanfordBootleg}%
  \cite[V]{c_design_thinking_online_nonpeer_dschool_stanford};
    \enquote{Prototype as if your know you're right, but test as if you know
      you're wrong}. This phase allows you to gather feedback, refine
    solutions and further deepen the empathy work. The core of the process is
    an iterative sequence where data, gathered by observing users  interacting
    with a prototype in the appropriate context, is used to further refine the
    design- and problem-space.

\subsection{Design thinking and this \{...\} (project?)}

  \TODO[]{Figure the correct word for '<this current document>'}
  \TODO[]{How is design thinking used in this <project?{>}}

\subsection{Theories}
  \TODO[]{Finish Theories}
\subsection{Tools}
  \TODO[]{Finish Tools}
\subsection{Threats to validity}
  \TODO[]{Finish Threats to validity}

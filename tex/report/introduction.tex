%Communication and effectiveness have always been valued in industry,
%especially now in the modern age {\TODOCite}. \\

For anyone that has worked on a project of any type and size, it is known that
good and clear communication is vital for a projects success. In a 2006 study, apply
titled, \enquote{Early warning signs of it project failure: The dominant
dozen}\cite{c_warning_signs_project_article}, the authors draw from a panel of
19 experts and the post-mortems of 55 IT-projects in order to rank the most
important early warning signs for detecting a failing project. For this
paper, the interesting part is that signs related to communication are mentioned
in two of the top ten positions, namely, position three and nine respectively.
\\

While discussing what could be improved within the overall working-process one
story about deadlines and communication stood out. The team was faced with a
looming deadline, which is not a uncommon occurrence in the game development
business, together with added external pressure due to a unmovable date. And
though the task was performed in time, several people commented that the
process would have gone a lot smoother, they believed, if communications and
planning had stayed within the assigned software suit, designed for this
purpose. Instead, parts of the team left the digital communication channels and
reverted to using more hands-on systems, such as post-its and whiteboards in
order to spread information and planning decisions.




%Good and clear communication is vital to projects of all types and sizes.
%A 2006 study, titled
%\citetitle{c_warning_signs_project_article}\cite{c_warning_signs_project_article}
%asking 19 expert and sampling post-mortems from 55 different IT-project places
%two items related to communication in their top-10 list of \enquote{early
%warning signs} for
%IT-projects.

\subsection{Issues and aim}

  \TODO[]{Finish Issues and aim}

  This paper aims to figure out the answer to the following questions:

  \begin{enumerate}
    \item{
        \textit{
          What made the people switch off from the designated digital tooling?
        }
      }
    \item{
        \textit{
          What was seen as the most preferable aspect of the lowfi, hands-on
          approach that made it more appealing?
        }
      }
    \item{
        \textit{
          What knowledge can be drawn from cognition science and interface
          design in order to prevent this from happening again?
        }
      }
    \item{
        \textit{
          What are the smallest changes, based on the previous question, that
          can be made to most effectively create a measurable impact in
          perceived usability?
        }
      }
  \end{enumerate}

  Additionally, the aim is to identify at least two commonly performed
  work-flows that are perceived to be bottlenecks in the overall process and
  present a prototype, that incorporates solutions from the answer to question
  three and four in order to make these tasks less likely to impact
  overall communication in the future.

  \newpage

%  Why did communication break down?
%
%  According to the initial survey, there are rough spots that could be
%  improved.
%
%  What is the smallest changes based in cognition science and user interface
%  design that results in significant/measurable changes in productivity and user
%  satisfaction.
%  (Define "significant")

\subsection{Scope and limitations}

  \TODO[]{Finish Scope and limitation}

  Initially, there were many interesting candidates regarding looking at
  different interactions among the group of software products used in the team.
  However, since there is a upper time limit to this work,
  Shotgun\cite{c_product_shotgun} was chosen as the software that should be
  analyzed more in depth. \\

  Since the group of 25 people interacted with during this work is rather small
  seen in the context of statistic sampling, it is not possible to
  derive any statistically significant conclusions as a result of this work.
  Instead the data will be evaluated with this in mind and used to guide the
  direction for possible future, more extensive investigation of this topic
  within game development, or, alternatively more general software development.\\


%  Since there is an upper limit to the time

%  25 people, small sample.
%
%  There are many things that can be improved in order to make it run better.
%  (automation/autofill etc.)
%
%  Identify four work-processes, one for each persona, that can be measured
%  using the established framework, and the try to improve this with as few
%  modifications as possible.
%
%  Finish with a mock-up/low fidelity version of the modified software to
%  continue building from.


\subsection{Report structure}

  \TODO[]{Finish Report structure}

  Introduction \\
  Methodology \\
  Results \\
  Discussions and conclusions \\
  Appendices \\

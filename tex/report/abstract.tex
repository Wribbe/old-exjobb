% Abstract.
%----------

Working as a team in the gaming industry can be challenging as is, and it
does not get better with the degradation of effectiveness and communication
that potential stress of looming deadlines brings with it. \\

This thesis examines if it is possible to counter the reduction
in communication and effectiveness by making it easier to work with the
available tooling. More specifically, it tries to answer the question; is it
possible to reduce the negative effects of stress through tailor-made
web-interfaces interfacing with available tooling that utilise findings from
Cognitive Science and Interface Design?  \\

The main method used in the study is design thinking, which is composed the
following steps: empathize, define, ideate, prototype and test. During the
empathize step interviews and surveys are conducted in order to get a measure
of the how the users currently view working with {\shotgun}.
Using the initial data, personas representing the requirements and needs of
identified user-groups and stakeholders are created.
After the define and ideate stages, manipulable prototypes are cerated using
{\python} and {\flask}. The prototypes are then tested and monitored using
{\recordingSystem} together with {\reviewProtocol} with the results
feeding back into the prototype iteration cycle for a total of {\numIterations}
iterations. \\

By analyzing the observation data gathered using {\recordingSystem} the
conclusion that {\texttt{<insert conclusion here>}} is (not) possible through
usage of them method described above. \\

\textbf{Keywords:}
  Interaction Design,
  Tooling,
  Stress,
  Development,
  Design-Thinking.
